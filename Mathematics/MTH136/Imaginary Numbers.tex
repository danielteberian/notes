\documentclass{article}
\usepackage{amsmath}
\begin{document}
\title{Imaginary Numbers}
\author{Daniel Teberian}
\date{March 2, 2023{}}
\maketitle{}

\paragraph{Representing a value with a variable is a concept that one learns very early on in studying algebra. However, what if we need to assign a number to a variable, but the number is not real? As is the case with the word itself, a number that is not real is, in fact, an imaginary number.}

\paragraph{But what is an imaginary number, and in what circumstance would such a number come up? Being a novice in the area of mathematics (rather, I am experienced, but not very skilled in the field), I was similarly puzzled at the idea of there being such a thing as an imaginary number. After all, if a number does not exist, why bother mentioning said number at all? If the number was so important, why did it not become a real number, rather than forcing countless students to learn how to work with such a seemingly silly number? I cannot say that I fully grasp the reason for needing imaginary numbers, but I do know how such numbers function.}

\paragraph{Let us look at an example of an expression that seems fairly simple upon first glance:}
$x^2$ = -1

\paragraph{Did you see an issue with the expression? If you missed it, try to solve the expression. What number could be squared, in order to equal -1?}

\end{document}
